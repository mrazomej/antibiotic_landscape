% Options for packages loaded elsewhere
\PassOptionsToPackage{unicode}{hyperref}
\PassOptionsToPackage{hyphens}{url}
\PassOptionsToPackage{dvipsnames,svgnames,x11names}{xcolor}
%
\documentclass[
]{scrartcl}

\usepackage{amsmath,amssymb}
\usepackage{iftex}
\ifPDFTeX
  \usepackage[T1]{fontenc}
  \usepackage[utf8]{inputenc}
  \usepackage{textcomp} % provide euro and other symbols
\else % if luatex or xetex
  \usepackage{unicode-math}
  \defaultfontfeatures{Scale=MatchLowercase}
  \defaultfontfeatures[\rmfamily]{Ligatures=TeX,Scale=1}
\fi
\usepackage{lmodern}
\ifPDFTeX\else  
    % xetex/luatex font selection
\fi
% Use upquote if available, for straight quotes in verbatim environments
\IfFileExists{upquote.sty}{\usepackage{upquote}}{}
\IfFileExists{microtype.sty}{% use microtype if available
  \usepackage[]{microtype}
  \UseMicrotypeSet[protrusion]{basicmath} % disable protrusion for tt fonts
}{}
\makeatletter
\@ifundefined{KOMAClassName}{% if non-KOMA class
  \IfFileExists{parskip.sty}{%
    \usepackage{parskip}
  }{% else
    \setlength{\parindent}{0pt}
    \setlength{\parskip}{6pt plus 2pt minus 1pt}}
}{% if KOMA class
  \KOMAoptions{parskip=half}}
\makeatother
\usepackage{xcolor}
\setlength{\emergencystretch}{3em} % prevent overfull lines
\setcounter{secnumdepth}{5}
% Make \paragraph and \subparagraph free-standing
\ifx\paragraph\undefined\else
  \let\oldparagraph\paragraph
  \renewcommand{\paragraph}[1]{\oldparagraph{#1}\mbox{}}
\fi
\ifx\subparagraph\undefined\else
  \let\oldsubparagraph\subparagraph
  \renewcommand{\subparagraph}[1]{\oldsubparagraph{#1}\mbox{}}
\fi


\providecommand{\tightlist}{%
  \setlength{\itemsep}{0pt}\setlength{\parskip}{0pt}}\usepackage{longtable,booktabs,array}
\usepackage{calc} % for calculating minipage widths
% Correct order of tables after \paragraph or \subparagraph
\usepackage{etoolbox}
\makeatletter
\patchcmd\longtable{\par}{\if@noskipsec\mbox{}\fi\par}{}{}
\makeatother
% Allow footnotes in longtable head/foot
\IfFileExists{footnotehyper.sty}{\usepackage{footnotehyper}}{\usepackage{footnote}}
\makesavenoteenv{longtable}
\usepackage{graphicx}
\makeatletter
\def\maxwidth{\ifdim\Gin@nat@width>\linewidth\linewidth\else\Gin@nat@width\fi}
\def\maxheight{\ifdim\Gin@nat@height>\textheight\textheight\else\Gin@nat@height\fi}
\makeatother
% Scale images if necessary, so that they will not overflow the page
% margins by default, and it is still possible to overwrite the defaults
% using explicit options in \includegraphics[width, height, ...]{}
\setkeys{Gin}{width=\maxwidth,height=\maxheight,keepaspectratio}
% Set default figure placement to htbp
\makeatletter
\def\fps@figure{htbp}
\makeatother

% Change font size
\usepackage[font={footnotesize, sf}, labelfont=bf, margin=0.05\linewidth]{caption}
\renewcommand{\familydefault}{\sfdefault}
% Bold the 'Figure #' in the caption and separate it from the title/caption
% with a period
% Captions will be left justified
\usepackage[
	aboveskip=1pt,
	labelfont=bf,
	labelsep=period,
	justification=raggedright,
	singlelinecheck=off
]{caption}
\renewcommand{\figurename}{Fig}

% Add numbered lines
\usepackage{lineno}
% \linenumbers

% Package to include multiple title pages
% This will allow me to add a tile to the main text and to the SI
\usepackage{titling}

% Define command to begin the supplementary section
\newcommand{\beginsupplement}{
\setcounter{page}{1} % Restart page counter
\setcounter{section}{0} % Restart section counter
% \renewcommand{\thesection}{\Alph{section}}%
\renewcommand{\thesection}{}
\renewcommand{\thesubsection}{\Alph{subsection}}
\setcounter{table}{0} % Restart table counter
\renewcommand{\thetable}{\Alph{table}}
\setcounter{figure}{0} % Restart figure counter
\renewcommand{\thefigure}{\Alph{figure}}%
\setcounter{equation}{0} % Restart equation counter
\renewcommand{\theequation}{S\arabic{equation}}%
}

% Add author affiliations
\usepackage{authblk}

% Allow to define personalized colors
\usepackage{xcolor}
\makeatletter
\@ifpackageloaded{caption}{}{\usepackage{caption}}
\AtBeginDocument{%
\ifdefined\contentsname
  \renewcommand*\contentsname{Table of contents}
\else
  \newcommand\contentsname{Table of contents}
\fi
\ifdefined\listfigurename
  \renewcommand*\listfigurename{List of Figures}
\else
  \newcommand\listfigurename{List of Figures}
\fi
\ifdefined\listtablename
  \renewcommand*\listtablename{List of Tables}
\else
  \newcommand\listtablename{List of Tables}
\fi
\ifdefined\figurename
  \renewcommand*\figurename{Figure}
\else
  \newcommand\figurename{Figure}
\fi
\ifdefined\tablename
  \renewcommand*\tablename{Table}
\else
  \newcommand\tablename{Table}
\fi
}
\@ifpackageloaded{float}{}{\usepackage{float}}
\floatstyle{ruled}
\@ifundefined{c@chapter}{\newfloat{codelisting}{h}{lop}}{\newfloat{codelisting}{h}{lop}[chapter]}
\floatname{codelisting}{Listing}
\newcommand*\listoflistings{\listof{codelisting}{List of Listings}}
\makeatother
\makeatletter
\makeatother
\makeatletter
\@ifpackageloaded{caption}{}{\usepackage{caption}}
\@ifpackageloaded{subcaption}{}{\usepackage{subcaption}}
\makeatother
\ifLuaTeX
  \usepackage{selnolig}  % disable illegal ligatures
\fi

%%%%%%%%%%%%%%%%%%%%%%%%%%%%%%%%%%%%%%%%%%%%%%%%%%%%%%%%%%%%%%%%%%%%%%%%%%%%%%%%
\usepackage[
    style=vancouver,
    sorting=none,				% Do not sort bibliography
	url=false, 					% Do not show url in reference
	doi=false, 					% Do not show doi in reference
	isbn=false, 				% Do not show isbn link in reference
	eprint=false, 			    % Do not show eprint link in reference
	maxbibnames=9, 			    % Include up to 9 names in citation
	firstinits=true,
]{biblatex}
%%%%%%%%%%%%%%%%%%%%%%%%%%%%%%%%%%%%%%%%%%%%%%%%%%%%%%%%%%%%%%%%%%%%%%%%%%%%%%%%
\addbibresource{references.bib}
\IfFileExists{bookmark.sty}{\usepackage{bookmark}}{\usepackage{hyperref}}
\IfFileExists{xurl.sty}{\usepackage{xurl}}{} % add URL line breaks if available
\urlstyle{same} % disable monospaced font for URLs
\hypersetup{
  pdftitle={Rapid evolution follows predictable paths on learnable low-dimensional manifolds},
  pdfauthor={Manuel Razo-Mejia; Madhav Mani; Dmitri Petrov},
  pdfkeywords={variational autoencoders, evolutionary
dynamics, antibiotic resistance},
  colorlinks=true,
  linkcolor={blue},
  filecolor={Maroon},
  citecolor={Blue},
  urlcolor={Blue},
  pdfcreator={LaTeX via pandoc}}

%%%%%%%%%%%%%%%%%%%%%%%%%%%%%%%%%%%%%%%%%%%%%%%%%%%%%%%%%%%%%%%%%%%%%%%%%%%%%%%%
% Add title
\title{Rapid evolution follows predictable paths on learnable
low-dimensional manifolds}
%%%%%%%%%%%%%%%%%%%%%%%%%%%%%%%%%%%%%%%%%%%%%%%%%%%%%%%%%%%%%%%%%%%%%%%%%%%%%%%%

%%%%%%%%%%%%%%%%%%%%%%%%%%%%%%%%%%%%%%%%%%%%%%%%%%%%%%%%%%%%%%%%%%%%%%%%%%%%%%%%
% Add list of authors

% Loop through authors
\author[1,*]{Manuel Razo-Mejia}
\author[3,4]{Madhav Mani}
\author[1,2,5]{Dmitri Petrov}

% Loop through affiliations
\affil[1]{Department of Biology, Stanford University}
\affil[2]{Stanford Cancer Institute, Stanford University School of
Medicine}
\affil[3]{NSF-Simons Center for Quantitative Biology, Northwestern
University}
\affil[4]{Department of Engineering Sciences and Applied Mathematics,
Northwestern University}
\affil[5]{Chan Zuckerberg Biohub}

% Add corresponding author
%%%%%%%%%%%%%%%%%%%%%%%%%%%%%%%%%%%%%%%%%%%%%%%%%%%%%%%%%%%%%%%%%%%%%%%%%%%%%%%%

%%%%%%%%%%%%%%%%%%%%%%%%%%%%%%%%%%%%%%%%%%%%%%%%%%%%%%%%%%%%%%%%%%%%%%%%%%%%%%%%
% Set affiliations in small font
\renewcommand\Affilfont{\itshape\small}
%%%%%%%%%%%%%%%%%%%%%%%%%%%%%%%%%%%%%%%%%%%%%%%%%%%%%%%%%%%%%%%%%%%%%%%%%%%%%%%%

%%%%%%%%%%%%%%%%%%%%%%%%%%%%%%%%%%%%%%%%%%%%%%%%%%%%%%%%%%%%%%%%%%%%%%%%%%%%%%%%
% Remove date
\date{}
%%%%%%%%%%%%%%%%%%%%%%%%%%%%%%%%%%%%%%%%%%%%%%%%%%%%%%%%%%%%%%%%%%%%%%%%%%%%%%%%

\begin{document}
\maketitle
\begin{abstract}
TBD
\end{abstract}
% Remove main text from the table of contents by specifying not to include
% any section or subsection
\addtocontents{toc}{\protect\setcounter{tocdepth}{-1}}

% Define reference segment for main text
\begin{refsegment}
% Generate filter to not include references from main text in the
% supplemental references
\defbibfilter{notother}{not segment=\therefsegment}

\section{Introduction}\label{introduction}

Over the last decade, evolutionary biology has begun a transition from a
primarily descriptive field aimed at understanding the history of life
on Earth to a predictive science forecasting and controlling adaptive
trajectories of evolving populations \autocite{Lassig2017,lassig2023a}.
Such predictions have had real-world consequences, from viral dynamics
\autocite{uksza2014} to cancer immunotherapy \autocite{luksza2017}.
However, it is still unclear to what extent evolutionary trajectories
under different environmental pressures can be predicted. Two main
challenges make this question difficult to answer. First, the vast
number of genetic changes available to a population makes it impossible
to exhaustively sample and measure the phenotypic consequences of each
mutation \autocite{louis2016}. Second, for most organisms but a few
simple model systems, it is unknown what the phenotypic changes driving
adaptation are {[}\emph{cite?}{]}.

Nevertheless, an extensive body of empirical studies and theoretical
work has suggested that the dynamics of adaptation are highly
constrained, making them potentially predictable. From examples of
enormous redundancy in the genotype-phenotype map, making the same
phenotype accessible through many different genetic changes
\autocite{wagner2008,louis2016}, to examples of a limited number of
viable adaptive trajectories \autocite{weinreich2006}, reducing the
number of sequential changes that can lead to adaptation, the evidence
is mounting that there are general principles governing the dynamics of
adaptation. Moreover, recent theoretical work on the evolution of
complex gene regulatory networks has shown that the output space
available to these networks to adapt to new tasks is highly constrained
\autocite{kaneko2015,furusawa2018,sato2020,sato2023}.

Inspired by this body of work, we set out to develop a general
computational framework to assess the predictability of evolutionary
trajectories. We consider three main ingredients for a computational
method to tackle this question: 1. The method should be able to take
advantage of the data-rich landscape of modern biology, where
high-throughput fitness measurements are becoming increasingly common
\autocite{kinsler2020,maeda2020,iwasawa2022}. 2. It should be able to
uncover the potentially hidden simplicity in the structure of the
high-dimensional datasets generated by these experiments by projecting
them in a principled way into a lower-dimensional space visualizable by
humans. 3. The generated low-dimensional map should be accompanied by
the corresponding geometric information on how the high-dimensional
space is folded into the map, making the distances on the map meaningful
in the high-dimensional space.

Here, we present a computational method that fulfills these three
requirements. By implementing a geometry-aware variational
autoencoder---a neural network architecture that embeds high-dimensional
data into a low-dimensional space while simultaneously learning the
geometric transformations that map the learned low-dimensional space
back into the high-dimensional space---and applying it to data from a
recent study on the evolution of cross-resistance to different
antibiotics in \emph{E. coli} \autocite{iwasawa2022}, we show that
adaptive trajectories in the learned low-dimensional space are highly
predictable.

\section{Results}\label{results}

\section{Discussion}\label{discussion}

\section*{Materials and Methods}\label{materials-and-methods}
\addcontentsline{toc}{section}{Materials and Methods}

\section*{Acknowledgements}\label{acknowledgements}
\addcontentsline{toc}{section}{Acknowledgements}

We would like to thank Griffin Chure and Enrique Amaya for their helpful
advice and discussion. We are extremely grateful to Junichiro Iwasawa
for sharing the raw data from his study; his openness and willingness to
share data and the immediate response are examples of the best practices
in open science.

% Print main text references
\printbibliography[segment=\therefsegment]
% Close reference segment
\end{refsegment}

\clearpage



\end{document}
